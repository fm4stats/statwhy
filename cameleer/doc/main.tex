\documentclass{easychair}

\usepackage[utf8]{inputenc}
\usepackage[T1]{fontenc}
\usepackage{xspace}
\usepackage{stmaryrd}
\usepackage{./why3lang}
\usepackage{./gospel}
\usepackage{ifthen}
\usepackage[nomath,not1,notext]{stix}

\usepackage{hyperref}

\newcommand{\ity}[9]{#1\cdot#2\cdot#3\cdot#4\vdash #5:#6\cdot#7\cdot#8\cdot#9}

\newcommand{\whyml}{\textsf{WhyML}\xspace}
\newcommand{\kidml}{\textsf{KidML}\xspace}
\newcommand{\ocaml}{\textsf{OCaml}\xspace}
\newcommand{\dafny}{\textsf{Dafny}\xspace}
\newcommand{\mixins}{{Mixins}\xspace}
\newcommand{\coq}{\textsf{Coq}\xspace}
\newcommand{\clang}{{C}\xspace}
\newcommand{\cfml}{\textsf{CFML}\xspace}
\newcommand{\vocal}{\textsf{VOCaL}\xspace}
\newcommand{\osl}{\textsf{OSL}\xspace}
\newcommand{\why}{\textsf{Why3}\xspace}
\newcommand{\focalize}{\textsf{FoCaLiZe}\xspace}
\newcommand{\viper}{\textsf{Viper}\xspace}

\definecolor{thegray}{rgb}{0.9,0.9,0.9}
\definecolor{colorspec}{rgb}{0,0,0.797}
\definecolor{thered}{rgb}{0.797,0,0}
\definecolor{darkgreen}{rgb}{0.797,0,0}
\definecolor{theblue}{rgb}{0,0,0.797}
%\definecolor{thegray}{rgb}{0.949,0.949,0.949}
\definecolor{darkgray}{rgb}{0.8477,0.8477,0.8477}
\definecolor{ocaml-bg}{rgb}{0.9,0.9,0.9}
\definecolor{thegraygray}{rgb}{0.5,0.5,0.5}

% \DeclareCaptionFormat{listing}{\hfill#3}
% \captionsetup[lstlisting]{format=listing,singlelinecheck=false, margin=0pt,
%   font={sf,it,footnotesize},labelsep=space,labelfont=bf,belowskip=-1pt}

\newenvironment{judge}[2][c]
{\begin{minipage}[#1]{#2}
\hrule height2pt
\centerline{\vrule width2pt height 5pt\hfill \vrule width2pt height 5pt}
\begin{minipage}{\dimexpr\textwidth-4pt-1em}}
{\end{minipage}
\centerline{\vrule width2pt height 5pt\hfill \vrule width2pt height 5pt}
\hrule height2pt
\end{minipage}}

\newcommand{\todoi}[1]{\todo[inline]{#1}}

%%% syntax %%%
\newcommand{\taubar}{\overline{\tau}}
\newcommand{\btau}{\beta\tau}
\newcommand{\betaxbar}{\overline{\beta x}}
\newcommand{\betapxbar}{\overline{\beta' x}}
\newcommand{\btaubar}{\overline{\btau}}
\newcommand{\bitaubar}[1]{\overline{\beta_{#1}\tau}}
\newcommand{\bitauibar}[2]{\overline{\beta_{#1}\tau_{#2}}}
\newcommand{\bibar}[1]{\overline{\beta_{#1}}}
\newcommand{\bibarp}[1]{\overline{\beta_{#1}'}}
\newcommand{\bibarpp}[1]{\overline{\beta_{#1}''}}
\newcommand{\btaubarp}{\overline{\btau'}}
\newcommand{\bptaubar}{\overline{\beta'\tau}}
\newcommand{\bpptaubar}{\overline{\beta''\tau}}
\newcommand{\betabar}{\overline{\beta}}
\newcommand{\betabarp}{\overline{\beta'}}
\newcommand{\betabarpp}{\overline{\beta''}}
\newcommand{\betaibar}{\overline{\beta_i}}
\newcommand{\bitau}{\overline{\beta_i\tau_i}}
\newcommand{\bitaup}{\overline{\beta_i'\tau_i}}
\newcommand{\pibar}{\overline{\pi}}
\newcommand{\xbar}{\bar{x}}
\newcommand{\pibari}[1]{\overline{\pi_{#1}}}
\newcommand{\xbart}[1]{\overline{x:#1}}
\newcommand{\xpibar}{\overline{x:\pi}}
\newcommand{\pebar}{\overline{\bar{p}\Rightarrow e}}
\newcommand{\pbare}{\bar{p}\Rightarrow e}
\newcommand{\eebar}{\overline{E\Rightarrow e}}
\newcommand{\eepbar}{\overline{E'\abar\Rightarrow e'}}
\newcommand{\eabar}{\overline{E\abar\Rightarrow e'}}
\newcommand{\exbar}{\overline{E\xbar\Rightarrow e}}
\newcommand{\exbarp}{\overline{E\xbar\Rightarrow e'}}
\newcommand{\abar}{\bar{a}}
\newcommand{\expbar}{\bar{e}}
\newcommand{\pbar}{\bar{p}}
\newcommand{\vbar}{\bar{\nu}}
\newcommand{\ebar}{\overline{E}}
\newcommand{\nubar}{\bar{\nu}}
\newcommand{\alabel}[1]{\quad\text{#1}}
\newcommand{\sep}{\;\mid\;}
\newcommand{\record}[2]{
  \{\:\beta#1: #2\; ;\;\ldots\;;\; \beta#1: #2\:\}_{\overline{\alpha}}}
\newcommand{\alphabar}{\overline{\alpha}}
\newcommand{\canbeta}[1]{\betabar\:\lozenge\:#1}
\newcommand{\fun}[3]{\lambda x:#1\:#2.\: #3}
\newcommand{\funreg}{\fun{\reg}{\tau}{e}}
\newcommand{\funghost}{\fun{\ghost}{\tau}{e}}

%%% semantics %%%
\newcommand{\state}{\delta\cdot\mu}
\newcommand{\statep}{\mu'}
\newcommand{\statepp}{\mu''}
\newcommand{\stated}[1]{\delta[#1]\cdot\mu}
\newcommand{\murule}[1]{\state\cdot #1}
\newcommand{\muprule}[1]{\statep\cdot #1}
\newcommand{\mupprule}[1]{\statepp\cdot #1}
\newcommand{\trans}[2]{#1\Downarrow #2}
\newcommand{\transco}[2]{#1\Downarrow^{\textsf{co}} #2}
\newcommand{\cxt}[2]{\trans{\murule{#1}}{\muprule{#2}}}
\newcommand{\cxtp}[2]{\trans{\murule{#1}}{\mupprule{#2}}}
\newcommand{\cxtco}[2]{\transco{\murule{#1}}{\muprule{#2}}}
\newcommand{\cxtcop}[2]{\transco{\murule{#1}}{\mupprule{#2}}}
\newcommand{\cxtcod}[1]{\transco{\murule{#1}}{\diverges}}
\newcommand{\cxtcodp}[1]{\transco{\muprule{#1}}{\diverges}}
\newcommand{\abort}{\texttt{abort}}

%%% progress %%%
\newcommand{\progdeltamu}[1]{#1\downarrow}
\newcommand{\prog}[1]{\progdeltamu{#1}}
\newcommand{\progrule}[2]{\inferrule*[Right=(#1)]{ }{\prog{#2}}}

%%% typing %%%
\newcommand{\dgste}[5]{#1\cdot #2\cdot #3\vdash #4 : #5}
\newcommand{\dgte}[4]{\dgste{#1}{#2}{\Sigma}{#3}{#4}}
\newcommand{\gte}[3]{\dgste{\Delta}{#1}{\Sigma}{#2}{#3}}
\newcommand{\nogte}[2]{\dgste{\Delta}{\emptyset}{\Sigma}{#1}{#2}}
\newcommand{\nogste}[3]{\dgste{\Delta}{\emptyset}{#1}{#2}{#3}}
\newcommand{\gste}[4]{\dgste{\Delta}{#1}{#2}{#3}{#4}}
\newcommand{\dte}[3]{\dgste{#1}{\Gamma}{\Sigma}{#2}{#3}}
\newcommand{\te}[2]{\dgte{\Delta}{\Gamma}{#1}{#2}}
\newcommand{\tp}[2]{\vdash #1 : #2}
\newcommand{\ori}[2]{\underset{#1}{\bigsqcup}#2}
\newcommand{\disji}[2]{\underset{#1}{\bigcup}#2}
\renewcommand{\sup}[2]{#1\sqcup#2}
\newcommand{\reg}{\texttt{reg}}
\newcommand{\ghost}{\texttt{ghost}}
\newcommand{\absurd}{\mathtt{absurd}}
\newcommand{\raises}{{\tt raises}}
\newcommand{\unit}{{\tt Unit}}
\newcommand{\bool}{{\tt Bool}}
\newcommand{\diverges}{{\tt div}}
\newcommand{\try}{{\tt try}}
\newcommand{\with}{{\tt with}}
\newcommand{\qqquad}{\qquad\qquad\quad}
\newcommand{\poly}{\langle\alphabar\rangle}
% \newcommand{\inst}{\langle\taubar\rangle}

%%% extraction %%%
\newcommand{\extarg}[2]{\mathcal{E}_{#1}(#2)}
\newcommand{\extbeta}[1]{\extarg{\betabar}{#1}}
\newcommand{\extkappa}[1]{\extarg{\kappa}{#1}}
\newcommand{\extsigma}[1]{\extarg{\sigma}{#1}}
\newcommand{\extkappap}[1]{\extarg{\kappa'}{#1}}
\newcommand{\ext}[1]{\extarg{}{#1}}
\newcommand{\extghost}[1]{\extarg{\ghost}{#1}}
\newcommand{\extreg}[1]{\extarg{\reg}{#1}}
\newcommand{\masksub}[2]{#1\sqsubseteq#2}
\newcommand{\notmasksub}[2]{#1\not\sqsubseteq#2}
\newcommand{\masklt}[2]{#1\sqsubset#2} \newcommand{\exttry}[1]{
  \overline{E\xbar_{\restriction{#1}}\Rightarrow\extbeta{e'}}}
\newcommand{\extp}{\overline{\ext{\pbar}\Rightarrow\extsigma{e}}}
\newcommand{\extrec}[2]{\ext{ \{\:\beta_1#1_1: #2_1\; ;\;\ldots\;;\;
    \beta_n#1_n: #2_n\:\}}} \newcommand{\extrecreg}[2]{ \{\:\overline{{\tt
      reg}\:#1_i : #2_i}\:\}_{\alphabar}} \newcommand{\mask}[1]{\mathcal{M}(#1)}
\newcommand{\exttyp}[1]{
  \dgste{\mathcal{E}(\Delta)}{\mathcal{E}(\Gamma)}{\Sigma}
  {\mathcal{E}_{\betabar}(e)}{\mathcal{E}_{\betabar}(#1)}}
\renewcommand{\restriction}[1]{\mathord{\upharpoonright}#1}
\newcommand{\extenv}{\ext{\Delta\cdot\Gamma\cdot\Sigma}}
\newcommand{\extte}[2]{\extenv\vdash #1 : #2}
\newcommand{\extdgste}[5]{\ext{#1}\cdot\ext{#2}\cdot\ext{#3}\vdash #4 : #5}
\newcommand{\cxtext}[2]{\ext{\delta}\cdot\ext{\mu}\cdot #1\Downarrow #2}
\newcommand{\cxtpext}[2]{\ext{\delta}\cdot\ext{\mu'}\cdot #1\Downarrow #2}
\newcommand{\cxtppext}[2]{\ext{\delta}\cdot\ext{\mu''}\cdot #1\Downarrow #2}
\newcommand{\cxtcodext}[1]{\ext{\delta}\cdot\ext{\mu}\cdot
  #1\Downarrow^{\textsf{co}}\diverges}
\newcommand{\cxtcodextp}[1]{\ext{\delta}\cdot\ext{\mu'}\cdot
  #1\Downarrow^{\textsf{co}}\diverges}

%%% defs & theorems %%%
% \newtheorem{theorem}{Theorem}[section]
% \newtheorem{lemma}[theorem]{Lemma}
% \newtheorem{prop}[theorem]{Property}
% \newtheorem{corollary}{Corollary}[theorem]
% \theoremstyle{definition}
% \newtheorem{definition}{Definition}[section]

% \newcommand{\case}[1]{\paragraph{case \color{blue}{(\textsc{#1}).}}}
% \newcommand{\rulename}[1]{{\color{blue}{\textsc{(\textbf{#1})}}}}

\usepackage{./mathpartir}

\newboolean{longversion}
\setboolean{longversion}{false} % toggle to show or hide comments

\usepackage{tikz}
\usetikzlibrary{arrows, shapes.misc, positioning}
\tikzset{>=stealth'}

\definecolor{thegray}{rgb}{0.9,0.9,0.9}
\definecolor{colorspec}{rgb}{0,0,0.797}
\definecolor{thered}{rgb}{0.797,0,0}
\definecolor{darkgreen}{rgb}{0.797,0,0}
\definecolor{theblue}{rgb}{0,0,0.797}
%\definecolor{thegray}{rgb}{0.949,0.949,0.949}
\definecolor{darkgray}{rgb}{0.8477,0.8477,0.8477}
\definecolor{ocaml-bg}{rgb}{0.9,0.9,0.9}
\definecolor{thegraygray}{rgb}{0.5,0.5,0.5}

\newcommand{\cameleer}{\textsf{Cameleer}\xspace}
% \newcommand{\why}{{Why3}\xspace}
% \newcommand{\whyml}{{WhyML}\xspace}
% \newcommand{\ocaml}{{OCaml}\xspace}
% \newcommand{\vocal}{{VOCaL}\xspace}
% \newcommand{\dafny}{{Dafny}\xspace}
% \newcommand{\mixins}{{Mixins}\xspace}
% \newcommand{\coq}{{Coq}\xspace}
% \newcommand{\clang}{{C}\xspace}
% \newcommand{\cfml}{{CFML}\xspace}
\newcommand{\GOSPEL}{{\textsf{GOSPEL}}\xspace}
\newcommand{\pre}{precondition\xspace}
\newcommand{\post}{postcondition\xspace}
\newcommand{\spec}{\mathcal{S}}
\newcommand{\loops}{\mathcal{I}}
\newcommand{\atts}{\mathcal{A}}
\newcommand{\kind}{\mathcal{K}}

\usepackage{mathtools}
\usepackage{tikz-cd}
\usetikzlibrary{decorations.pathmorphing}

% code from Gonzalo
\newcommand\xrsquigarrow[1]{%
  \mathrel{%
    \begin{tikzpicture}[%
      baseline={(current bounding box.south)}
      ]
      \node[%
      ,inner sep=.44ex
      ,align=center
      ] (tmp) {$\scriptstyle #1$};
      \path[%
      ,draw,<-
      ,decorate,decoration={%
        ,zigzag
        ,amplitude=0.7pt
        ,segment length=1.2mm,pre length=3.5pt
      }
      ]
      (tmp.south east) -- (tmp.south west);
    \end{tikzpicture}
  }
}
\newcommand\xlsquigarrow[1]{%
  \mathrel{%
    \begin{tikzpicture}[%
      ,baseline={(current bounding box.south)}
      ]
      \node[%
      ,inner sep=.44ex
      ,align=center
      ] (tmp) {$\scriptstyle #1$};
      \path[%
      ,draw,<-
      ,decorate,decoration={%
        ,zigzag
        ,amplitude=0.7pt
        ,segment length=1.2mm,pre length=3.5pt
      }
      ]
      (tmp.south west) -- (tmp.south east);
    \end{tikzpicture}
  }
}

\newcommand{\translate}[3]
{#1\xrsquigarrow{#2}#3}

\newcommand{\nbnote}[3]{
  % \fbox{\bfseries\sffamily\scriptsize#1}
  \fcolorbox{gray}{yellow}{\bfseries\sffamily\scriptsize#1}
  {\color{#2} \sffamily\small$\blacktriangleright$\textit{#3}$\blacktriangleleft$}
  % {\color{#2} \sffamily\small$\textit{#3}$}
  % \marginpar{\fbox{\bfseries\sffamily#1}}
}

\newcommand{\mario}[1]{\nbnote{Mario}{red}{#1}}
% \newcommand{\mario}[1]{}
\newcommand{\antonio}[1]{\nbnote{Antonio}{blue}{#1}} %

\newcommand{\myparagraph}[1]{\vspace{-12.5pt}\paragraph{#1}}
\newcommand{\mysection}[1]{\vspace{-2pt}\section{#1}}

\title{Draft: Translation from \ocaml to \whyml}

\author{}

\date{Working version of \today}

\institute{}

\pagenumbering{gobble}

\begin{document}

\maketitle

All the auxiliary functions and predicates are total
definitions. Fig.~\ref{fig:core-ocaml} and Fig.~\ref{fig:whyml} present the
definition of the selected \ocaml and \whyml subsets, respectively. On the
\whyml side, definition of~$t$ which stands for the logical subset of \whyml.
\cameleer will report a dedicated error message if a user tries to translate an
\ocaml program that syntactically falls out of the supported fragment.

\ifthenelse{\boolean{longversion}}
{
  These definitions are very close to the ASTs issued by the \ocaml
  and \whyml parsers. The \whyml formalization is also strongly inspired by the
  \kidml language, presented in Pereira's PhD
  thesis~\cite{parreirapereira:tel-01980343}.
}

\begin{figure}
  \centering
  % \begin{judge}[b]{\textwidth}
\[
  \begin{array}{llr}
    e & ::= \; x \sep \nu
        \sep \mathtt{if}\: e\: \mathtt{then}\: e\: \mathtt{else}\: e
        \sep \mathtt{match}\; e\; \mathtt{with}\;
        \overline{\talloblong\,p\Rightarrow e}
    & \alabel{Expressions}\\
      & \sep \mathtt{let} \: \atts \; x = e \; \atts \; \mathtt{in} \; e \\
      &  \sep \mathtt{let} \: \atts \; \mathtt{rec} \: f = e \; \atts \;
        \overline{\mathtt{and}\;f = e \; \atts} \; \mathtt{in} \; e
        \sep f\,\expbar \\
      & \sep \{ \overline{f = e} \} \sep e.f \sep e.f \leftarrow e \\
      & \sep \mathtt{raise} \; E\expbar
        \sep \mathtt{try} \; e \; \mathtt{with} \; \exbar \\
      & \sep \mathtt{while} \: e \: \mathtt{do} \: \atts \: e \: \mathtt{done}
        \sep \mathtt{assert\;false} \\[.5em]

    p & ::= \; \_ \sep x \sep \overline{p} \sep C (p)
        \sep p\: \talloblong\: p \sep p\;\mathtt{as}\;x \sep p : \tau
        \sep \mathtt{exception}\;p & \alabel{Patterns}\\[.5em]

    \nu & ::= \; n \sep \mathtt{true} \sep \mathtt{false} \sep
          \mathtt{fun}\:\atts\:(x\beta) \rightarrow e & \alabel{Values}\\[.5em]

    \beta & ::= \; \texttt{reg} \sep \texttt{ghost} & \alabel{Ghost attribute}
    \\[.5em]

    \tau & ::= \; \alpha \sep \tau\rightarrow\tau \sep \overline{\tau}
           \sep \overline{\tau}\:C & \alabel{Type expression} \\[.5em]

    \pi & ::= \; \beta\tau & \hspace{-40pt}
                             \alabel{Type with ghost status}\\[.5em]

    d & ::= \; \texttt{exception}\: E : \pibar
        \sep \texttt{type} \; td  \; \overline{\texttt{and}\; td}
    & \alabel{Top-level declarations} \\
      & \sep \mathtt{let}\: \atts \; f = e \: \atts
        \sep \mathtt{let}\: \atts \; \texttt{rec}\; f = e \: \atts\;
        \overline{\mathtt{and}\;f = e \; \atts}\\
      & \sep \texttt{module}\:\mathcal{M} = m\\[.5em]

    td & ::= \; \alphabar \: T
         \sep \alphabar \: T = \tau
         \sep \alphabar \: T = \{\: \overline{f : \pi}\:\}\: \atts
         \sep \alphabar \: T =
         \overline{\talloblong\,C\,\texttt{of}\,\overline{\tau}}
    & \alabel{Type definition}\\[.5em]

    m & ::= \; \texttt{struct}~~\overline{d}~~\texttt{end}
        \sep \texttt{functor}(\mathcal{X}: mt) \rightarrow m
            & \alabel{Modules} \\[.5em]

    mt & ::= \; \texttt{sig}\;\overline{s}\;\texttt{end} & \alabel{Module
                                                           types}\\[.5em]

    s & ::= \; \texttt{val}\:\atts\;f : \pi \: \atts
        \sep \texttt{type}\: td \; \overline{\texttt{and}\; td}
    & \alabel{Signatures}\\[.5em]

    p & ::= \; \overline{d} & \alabel{Program}
  \end{array}
\]
% \end{judge}
\caption{Syntax of core \ocaml.}
\label{fig:core-ocaml}
\end{figure}

\begin{figure}[h!]
  \centering
  % \begin{judge}[b]{\textwidth}
\[
  \begin{array}{llr}
    e & ::= \; x \sep \nu
        \sep \mathtt{if}\: e\: \mathtt{then}\: e\: \mathtt{else}\: e
        \sep \mathtt{match}\; e\; \mathtt{with}\;
        \overline{\talloblong\,p\Rightarrow e} \;
        \mathtt{end}
    & \alabel{Expressions}\\
      & \sep \mathtt{let} \; \kind\; \beta x = e \; \mathtt{in} \; e \\
      & \sep \mathtt{rec}\;\kind\;f(\overline{\beta x})\;\spec = e \;
        \overline{\mathtt{with}\;\kind\;f (\overline{\beta x}) = e \; \spec} \;
        \mathtt{in} \; e
        \sep f\,\expbar \\
      & \sep \{ \overline{f = e} \} \sep e.f \sep e.f \leftarrow e \\
      & \sep \mathtt{raise} \; E\expbar
        \sep \mathtt{try} \; e \; \mathtt{with} \; \exbar \; \mathtt{end} \\
      & \sep \mathtt{while} \: e \: \mathtt{do} \: \loops \: e \: \mathtt{done}
        \sep \mathtt{absurd}
        \sep \mathtt{ghost} \; e\\[.5em]

    p & ::= \; \_ \sep x \sep \overline{p} \sep C p
        \sep p\: \talloblong\: p \sep p\;\mathtt{as}\;x \sep p : \tau
        \sep \mathtt{exception}\;p & \alabel{Patterns}\\[.5em]

    \nu & ::= \; n \sep \mathtt{true} \sep \mathtt{false}
          \sep \mathtt{fun}\;\kind\; (\overline{\beta x}) \; \spec \rightarrow
          e & \alabel{Values} \\[.5em]

    \beta & ::= \; \texttt{reg} \sep \texttt{ghost} & \alabel{Ghost status}
    \\[.5em]

    \tau & ::= \; \alpha \sep \tau\rightarrow\tau \sep \overline{\tau}
           \sep C\,\overline{\tau} & \alabel{Type expression} \\[.5em]

    \pi & ::= \;\beta\tau & \hspace{-4pt}\alabel{Type with ghost status}\\[.5em]

    \kind & ::= \;\texttt{reg} \sep \texttt{logic} & \alabel{Function
                                                          kind}\\[.5em]

    \spec & ::= \; \texttt{requires}\; \overline{t}~~
             \texttt{ensures}\; \overline{t}~~
            \texttt{variant}\; \overline{t}
    & \alabel{Function specification} \\[.5em]

    \loops & ::= \; \texttt{invariant}\; \overline{t}~~\texttt{variant}\;
             \overline{t} & \alabel{Loop specification} \\[.5em]

    d & ::= \; \texttt{exception}\: E : \pibar
        \sep \texttt{type}\: td \; \overline{\texttt{with}\:td}~~~~~
                                  & \alabel{Top-level declarations} \\
      & \sep \mathtt{let} \; \mathcal{K} \; f = e \\
      & \sep \mathtt{let} \; \texttt{rec} \; \mathcal{K} \; f(\overline{\beta x})
        \:\spec = e \;
        \overline{\mathtt{with}\;\kind\;f (\overline{\beta x})\:\spec = e}\\
      & \sep \mathtt{val}\;\mathcal{K}\;\beta f(\overline{x:\pi})\:\spec : \pi
        \sep \texttt{scope} \;\mathcal{M}\;\overline{d} \; \texttt{end} \\[.5em]

    td & ::= \; T\alphabar
         \sep T\alphabar = \tau
         \sep
         T\alphabar = \{\: \overline{f : \pi}\:\}\:\texttt{invariant}\:\bar{t}
         \sep T\alphabar = \overline{\talloblong\,C\,\overline{\tau}}
    & \alabel{Type definition}\\[.5em]

    p & ::= \; \texttt{module}\;\mathcal{M}\;\overline{d}\;\texttt{end}
    & \alabel{Program}
  \end{array}
\]
% \end{judge}
\caption{Syntax of core \whyml.}
\label{fig:whyml}
\end{figure}

\paragraph{Expressions.} Selected \ocaml expressions include variables ($x$
ranges over program variables, while~$f$ is used for function names), the
conditional \texttt{if..then..else}, local bindings of (possibly recursive)
expressions, function application, records manipulation, treatment of
exceptions, loop construction, and finally the \texttt{assert false}
expression. Values include numerical and Boolean constants, as well as anonymous
functions where arguments are annotated with a ghost status. Recursive
definitions are limited to functions and application is limited to the
application of a function name to a list of arguments. The latter is just to
ease presentation; the former is due to recursive definitions in \whyml being
limited to functions. Finally, $\atts$ notation stands for a (possibly empty)
placeholder of \ocaml attributes, representing the original place in the
expression where \GOSPEL elements are introduced. Fig~\ref{fig:expressions}
presents the complete set of translation rules from \ocaml expressions into
\whyml.

% \mario{Não esquecer de referir o que significa o $\bar{t}$ $\Longrightarrow$
%   linguagem lógica de primeira ordem com extensões como polimorfismo, tipos
%   inductivos, ou ``suporte'' para elementos de ordem superior; no fundo, o
%   fragmento de especificação do \whyml.}

\begin{figure}
  \centering
\begin{judge}[b]{\textwidth}\footnotesize
  \[
    \begin{array}{c}
      \inferrule*[Left=(EAbsurd)]
      { }
      {\translate{\mathtt{assert\;false}}{expression}{\mathtt{absurd}}} \\[1em]

      \inferrule*[Left=(EFun)]
      {\mathtt{ghost}(\atts) = \beta \\
        \translate{\atts}{\textit{function spec}}{\mathcal{S}} \\
        \translate{e}{expression}{e'}}
      {\translate{\mathtt{fun}\;\atts\;x\rightarrow e}{expression}
      {\mathtt{fun}\;(\beta x)\;\mathcal{S} = e'}}\qquad

      \inferrule*[Right=(EApp)]
      {\translate{\overline{e}}{expression}{\overline{e'}}}
      {\translate{f\,\expbar}{expression}{f\,\overline{e'}}}\\[1em]

      \inferrule*[Lab=(ERec)]
      {
       \neg\textit{is\_ghost}(\atts_0) \\
       \mathit{kind}(\atts_0) = \kind \\
       \translate{\atts_1}{\textit{function spec}}{\spec_1} \\
       \translate{e_0}{function}{\overline{x\beta},e_0'} \\\\
       \translate{\overline{e_1}}{function}{\overline{\beta y},\overline{e_1'}} \\
       \translate{\overline{\atts_2}}{\textit{function spec}}{\overline{\spec_2}} \\
       \translate{e_2}{expression}{e_2'}}
      {\translate{\mathtt{let} \: \atts_0 \; \mathtt{rec} \: f_0 = e_0 \;
       \atts_1 \; \overline{\mathtt{and}\;f_1 = e_1 \; \atts_2} \;
       \mathtt{in} \; e_2}{expression}
       {\mathtt{rec}\;\kind\;f(\overline{\beta x})\;\spec_1 = e_0' \;
        \overline{\mathtt{with}\;\kind\;f_1 (\overline{\beta y}) = e_1' \;
        \spec_2}\;\mathtt{in} \; e_2'}}\\[1em]

      \inferrule*[Lab=(ERecGhost)]
      {\textit{is\_ghost}(\atts_0) \\
       \mathit{kind}(\atts_0) = \kind \\
       \translate{\atts_1}{\textit{function spec}}{\spec_1} \\
       \translate{e_0}{function}{\overline{x\beta},e_0'} \\\\
       \translate{\overline{e_1}}{function}{\overline{\beta y},\overline{e_1'}} \\
       \translate{\overline{\atts_2}}{\textit{function spec}}{\overline{\spec_2}} \\
       \translate{e_2}{expression}{e_2'}}
      {\translate{\mathtt{let} \: \atts_0 \; \mathtt{rec} \: f_0 = e_0 \;
       \atts_1 \; \overline{\mathtt{and}\;f_1 = e_1 \; \atts_2} \;
       \mathtt{in} \; e_2}{expression}
       {\mathtt{rec}\;\kind\;f(\overline{\beta x})\;\spec_1 = \texttt{ghost}\:e_0'
      \;\overline{\mathtt{with}\;\kind\;f_1 (\overline{\beta y}) = e_1' \;
        \spec_2}\;\mathtt{in} \; e_2'}}\\[1em]

      % \inferrule*[Left=(ERecGhost)]
      % { \textit{is\_ghost}(\atts) \\
      % \translate{\atts}{\textit{function spec}}{\spec} \\
      % \mathit{kind}(\atts) = \kind \\\\
      % \translate{e_0}{function}{\betaxbar,e_0'} \\
      % \translate{e_1}{expressions}{e_1'}}
      % {\translate{\mathtt{let}\:\atts\;\mathtt{rec}\:f = e_0\;\atts'\;
      % \mathtt{in} \; e_1}{expression}
      % {\mathtt{rec}\:\kind\:f(\betaxbar) \;\spec = \mathtt{ghost}\:e_0'\;
      % \mathtt{in} \;e_1'}} \\[1em]

      \inferrule*[Left=(ELet)]
      {\neg\textit{is\_ghost}(\atts) \\ \mathtt{kind}(\atts) = \kind \\
      \neg\textit{is\_functional}(e_0) \\\\
      \translate{e_0}{expression}{e_0'} \\ \translate{e_1}{expression}{e_1'}}
      {\translate{\mathtt{let}\:\atts\:x = e_0\;\atts'\;\mathtt{in}\;e_1}
      {expression}{\mathtt{let}\: \kind \: x = e_0'\;\mathtt{in} \;e_1'}}\\[1em]

      \inferrule*[Left=(ELetGhost)]
      { \textit{is\_ghost}(\atts) \\ \mathtt{kind}(\atts) = \kind \\
      \neg\textit{is\_functional}(e_0) \\\\
      \translate{e_0}{expression}{e_0'} \\ \translate{e_1}{expression}{e_1'}}
      {\translate{\mathtt{let}\:\atts\:x = e_0\;\atts'\;\mathtt{in}\;e_1}
      {expression}{\mathtt{let}\: \kind \: x = \mathtt{ghost}\:e_0'\;
      \mathtt{in} \;e_1'}} \\[1em]

      \inferrule*[Left=(ELetGhostFun)]
      { \textit{is\_ghost}(\atts) \\ \textit{kind}(\atts) = \kind \\
      \textit{is\_functional}(e_0) \\\\
      \translate{\atts'}{\textit{function spec}}{\spec} \\
      \translate{e_0}{function}{\betaxbar, e_0'} \\
      \translate{e_1}{expression}{e_1'}}
      {\translate{\mathtt{let}\:\atts\:f = e_0\;\atts'\;\mathtt{in}\;e_1}
      {expression}{\mathtt{let}\;\kind\;f =
      \mathtt{fun}(\betaxbar)\:\spec\rightarrow
      \mathtt{ghost}\:e_0'\;\mathtt{in} \;e_1'}} \\[1em]

      \inferrule*[Left=(ELetFun)]
      { \neg\textit{is\_ghost}(\atts) \\ \textit{kind}(\atts) = \kind \\
      \textit{is\_functional}(e_0) \\\\
      \translate{\atts'}{\textit{function spec}}{\spec} \\
      \translate{e_0}{function}{\overline{x\beta}, e_0'} \\
      \translate{e_1}{expression}{e_1'}}
      {\translate{\mathtt{let}\:\atts\:f = e_0\;\atts'\;\mathtt{in}\;e_1}
      {expression}{\mathtt{let}\;\kind\;f =
      \mathtt{fun}(\overline{\beta x})\:\spec\rightarrow e_0'\;\mathtt{in} \;e_1'}
      }\\[1em]

      \inferrule*[Left=(EWhile)]
      {\translate{\atts}{\textit{loop annotation}}{\loops} \\
      \translate{e_0}{expression}{e_0'} \\ \translate{e_1}{expression}{e_1'}}
      {\translate{\mathtt{while} \: e_0 \: \mathtt{do} \: \atts \: e_1 \:
      \mathtt{done}}{expression}
      {\mathtt{while} \: e_0' \: \mathtt{do} \: \loops \: e_1' \:
      \mathtt{done}}} \\[1em]

      \inferrule*[Left=(EIf)]
      {\translate{e_0}{expression}{e_0'} \\ \translate{e_1}{expression}{e_1'} \\
      \translate{e_2}{expression}{e_2'}}
      {\translate{\mathtt{if}\: e_0\: \mathtt{then}\: e_1\: \mathtt{else}\: e_2}
      {expression}
      {\mathtt{if}\: e_0'\: \mathtt{then}\: e_1'\: \mathtt{else}\: e_2'}}
      \\[1em]

      \inferrule*[Left=(EMatch)]
      {\translate{e_0}{expression}{e_0'} \\
      \translate{\overline{e_1}}{expression}{\overline{e_1'}}}
      {\translate{\mathtt{match}\; e_0\;
      \mathtt{with}\;\overline{\talloblong\,p\Rightarrow e_1}}
      {expression}
      {\mathtt{match}\; e_0'\;
      \mathtt{with}\;\overline{\talloblong\,p\Rightarrow e_1'}\,\texttt{end}}}

      \\[1em]

      \inferrule*[Left=(ERaise)]
      {\translate{\expbar}{expression}{\expbar'}}
      {\translate{\mathtt{raise} \; E\expbar}{expression}{\mathtt{raise} \;
        E\expbar'}}\qquad

      \inferrule*[Right=(ETry)]
      {\translate{e}{expression}{e'} \\ \translate{\exbar}{expression}{\exbar'}}
      {\translate{\mathtt{try} \; e \; \mathtt{with} \; \exbar}{expression}
      {\mathtt{try} \; e' \; \mathtt{with} \; \exbar'}\texttt{end}}
    \end{array}
  \]
\end{judge}
  \caption{Translation of \ocaml expressions into \whyml.}
  \label{fig:expressions}
\end{figure}

\paragraph{Top-level declarations.} Selected top-level declarations include
exceptions and type declaration, (mutually-recursive) function definition, and
introduction of sub-modules. An exception takes a list of~$\pi$ values, types
annotated with a ghost status, to account for the possibility of ghost
arguments. In \why vocabulary, this is the \emph{mask} of an exception. % The
% complete set of translation rules for top-level declarations and type
% definitions is given in Appendix~\ref{sec:transl-top-level}. We highlight here
% the cases of record type definition and sub-modules.
The attribute~$\atts$ after a record type definition is used to express in
\GOSPEL a \emph{type invariant}, \emph{i.e.}, a predicate that every inhabitant
of such type must satisfy. Type invariants are readily supported by \why.
%
% \[
%   \inferrule*[Left=(TDRecord)]
%   {\translate{\atts}{\textit{type invariant}}{\bar{t}}}
%   {\translate{\alphabar\:T = \{\: \overline{f : \pi}\:\}\:\atts}
%     {declaration}
%     {T\alphabar = \{\:\overline{f:\pi}\:\}\:\mathtt{invariant}\:\bar{t}}}
% \]
%
Each field of the record type is also annotated with a ghost
status. Fig~\ref{fig:top-level} and Fig.~\ref{fig:type-def} present,
respectively, translation rules for \ocaml top-level declarations and type
definitions.

\begin{figure}[h!]
  \centering
  \begin{judge}[b]{\textwidth}\footnotesize
    \[
      \begin{array}{c}
        \inferrule*[Left=(DModule)]
        {\translate{m}{module}{\overline{d}}}
        {\translate{\mathtt{module}\:\mathcal{M} = m}{declaration}
        {\mathtt{scope}\:\mathcal{M}\;\overline{d}\;\mathtt{end}}} \\[1em]

        \inferrule*[Left=(DLetGhost)]
        {\mathit{is\_ghost}(\atts) \\ \textit{kind}(\atts') = \mathcal{K} \\
        \textit{is\_functional}(e)\\\\
        \translate{\atts}{\textit{function spec}}{\mathcal{S}} \\
        \translate{e}{function}{\betaxbar, e'}}
        {\translate{\mathtt{let}\: \atts \; f = e \: \atts'}{declaration}
        {\mathtt{let}\:f\:\mathcal{K} =
        \mathtt{fun}(\betaxbar)\:\mathcal{S}\rightarrow
        \mathtt{ghost}\;e'}}\\[1em]

        \inferrule*[Left=(DLet)]
        {\neg\mathit{is\_ghost}(\atts) \\ \textit{kind}(\atts') = \mathcal{K} \\
        \textit{is\_functional}(e)\\\\
        \translate{\atts}{\textit{function spec}}{\mathcal{S}} \\
        \translate{e}{function}{\betaxbar, e'}
        }
        {\translate{\mathtt{let}\: \atts \; f = e \: \atts'}{declaration}
        {\mathtt{let}\:f\:\mathcal{K} =
        \mathtt{fun}(\betaxbar)\:\mathcal{S}\rightarrow\;e'}} \\[1em]

      \inferrule*[Lab=(DRec)]
      {
       \neg\textit{is\_ghost}(\atts_0) \\
       \mathit{kind}(\atts_0) = \kind \\
       \translate{\atts_1}{\textit{function spec}}{\spec_1} \\
       \translate{e_0}{function}{\overline{x\beta},e_0'} \\\\
       \translate{\overline{e_1}}{function}{\overline{y:\pi},\overline{e_1'}} \\
       \translate{\overline{\atts_2}}{\textit{function spec}}{\overline{\spec_2}}}
      {\translate{\mathtt{let} \: \atts_0 \; \mathtt{rec} \: f_0 = e_0 \;
       \atts_1 \; \overline{\mathtt{and}\;f_1 = e_1 \; \atts_2}}{declaration}
       {\mathtt{rec}\;\kind\;f(\overline{\beta x})\;\spec_1 = e_0' \;
        \overline{\mathtt{with}\;\kind\;f_1 (\overline{y:\pi}) = e_1' \;
        \spec_2}}}\\[1em]

      \inferrule*[Lab=(DRecGhost)]
      {\textit{is\_ghost}(\atts_0) \\
       \mathit{kind}(\atts_0) = \kind \\
       \translate{\atts_1}{\textit{function spec}}{\spec_1} \\
       \translate{e_0}{function}{\overline{x\beta},e_0'} \\\\
       \translate{\overline{e_1}}{function}{\overline{y:\pi},\overline{e_1'}} \\
       \translate{\overline{\atts_2}}{\textit{function spec}}{\overline{\spec_2}}}
      {\translate{\mathtt{let} \: \atts_0 \; \mathtt{rec} \: f_0 = e_0 \;
       \atts_1 \; \overline{\mathtt{and}\;f_1 = e_1 \; \atts_2}}{declaration}
       {\mathtt{rec}\;\kind\;f(\overline{\beta x})\;\spec_1 = \texttt{ghost}\:e_0'
      \;\overline{\mathtt{with}\;\kind\;f_1 (\overline{y:\pi}) = e_1' \;
        \spec_2}}}\\[1em]

        % \inferrule*[Left=(DRec)]
        % {\neg\mathit{is\_ghost}(\atts) \\ \textit{kind}(\atts) = \mathcal{K}
        % \\\\
        % \translate{\atts'}{\textit{function spec}}{\mathcal{S}} \\
        % \translate{e}{function}{\betaxbar, e'}
        % }
        % {\translate{\mathtt{let}\: \atts \; \mathtt{rec}\; f = e \:
        % \atts'}{declaration}
        % {\mathtt{let}\:\mathtt{rec}\:\mathcal{K}\:f(\betaxbar)\:\mathcal{S} =
        % e'}}\\[1em]

        % \inferrule*[Left=(DRecGhost)]
        % {\mathit{is\_ghost}(\atts) \\ \textit{kind}(\atts) = \mathcal{K} \\\\
        % \translate{\atts}{\textit{function spec}}{\mathcal{S}} \\
        % \translate{e}{function}{\betaxbar, e'}
        % }
        % {\translate{\mathtt{let}\: \atts \; \mathtt{rec}\; f = e \:
        % \atts'}{declaration}
        % {\mathtt{let}\:\mathtt{rec}\:\mathcal{K}\:f(\betaxbar)\:\mathcal{S} =
        % \mathtt{ghost}\;e'}} \\[1em]

        \inferrule*[Left=(DType)]
        {\translate{td_0}{\textit{type definition}}{td_0'} \\
         \translate{\overline{td_1}}
         {\textit{type definition}}{\overline{td_1'}}}
        {\translate{\texttt{type} \; td_0  \; \overline{\texttt{and}\; td_1}}
         {declaration}
         {\texttt{type}\: td_0' \; \overline{\texttt{with}\:td_1'}}}\\[1em]

        \inferrule*[Left=(DExn)]
        { }
        {\translate{\texttt{exception}\: E : \pibar}{declaration}{
        \texttt{exception}\: E : \pibar}}
      \end{array}
    \]
  \end{judge}
  \caption{Translation of \ocaml top-level declarations into \whyml.}
  \label{fig:top-level}
\end{figure}

\begin{figure}
  \centering
  \begin{judge}[b]{\textwidth}\small
    \[
      \begin{array}{c}
        \inferrule*[Left=(TDAbstract)]
        { }
        {\translate{\alphabar\:T}{\textit{type definition}}{T\alphabar}}\qquad~

        \inferrule*[Right=(TDAlias)]
        { }
        {\translate{\alphabar\:T = \tau}{\textit{type definition}}
         {T\alphabar = \tau}}\\[1em]

        \inferrule*[Left=(TDRecord)]
        {\translate{\atts}{\textit{type invariant}}{\bar{t}}}
        {\translate{\alphabar\:T = \{\: \overline{f : \pi}\:\}\:\atts}
        {declaration}
        {T\alphabar = \{\:\overline{f:\pi}\:\}\:\mathtt{invariant}\:\bar{t}}}
        \\[1em]

        \inferrule*[Left=(TDVariant)]
        { }
        {\translate{\alphabar\:T =
         \overline{\talloblong\,C\,\texttt{of}\,\overline{\tau}}}
         {\textit{type definition}}
         {T\alphabar = \overline{\talloblong\,C\,\overline{\tau}}}} \\[1em]
      \end{array}
    \]
  \end{judge}
  \caption{Translation of \ocaml type definitions into \whyml.}
  \label{fig:type-def}
\end{figure}

\paragraph{Modules.} Fig.~\ref{fig:mod} presents translation rules for \ocaml
module expressions and module types.

\begin{figure}[h!]
  \centering
  \begin{judge}[b]{\textwidth}\small
  \[
    \begin{array}{c}
      \inferrule*[Left=(MStruct)]
      {\translate{\overline{d}}{declaration}{\overline{d'}}}
      {\translate{\mathtt{struct}\:\overline{d}\:\mathtt{end}}{\textit{module}}
      {\overline{d'}}} \\[1em]

      \inferrule*[Left=(MFunctor)]
      {\translate{mt}{\textit{module type}}{\overline{d}} \\
      \translate{m}{module}{\overline{d'}}}
      {\translate{\mathtt{functor}(\mathcal{X}: mt)\rightarrow m}
      {\mathit{module}}
      {\mathtt{scope}\:\mathcal{X}\;\overline{d}\;\mathtt{end}\;\overline{d'}}}
      \\[1em]

      \inferrule*[Left=(MTSig)]
      {\translate{\overline{s}}{signature}{\overline{d}}}
      {\translate{\mathtt{sig}\:s\:\mathtt{end}}{\textit{module type}}
      {\overline{d}}}
    \end{array}
  \]
  \end{judge}
  \caption{Translation of \ocaml module expressions and module types into
    \whyml.}
  \label{fig:mod}
\end{figure}

\paragraph{Signatures.} Fig.~\ref{fig:sigs} presents translation rules for
\ocaml signatures.

\begin{figure}[h!]
  \centering
  \begin{judge}[b]{\textwidth}\small
  \[
    \begin{array}{c}
      \inferrule*[Left=(SVal)]
      {\neg\textit{is\_ghost}(\atts) \\ \textit{kind}(\atts) = \kind \\\\
      \translate{\atts'}{\textit{function spec}}{\spec} \\
      \translate{\pi,\atts'}{\textit{function args}}{\overline{x:\pi'},\pi_{res}}}
      {\translate{\mathtt{val}\: \atts \: f : \pi \: \atts'}{signature}
      {\mathtt{val}\:\kind\: f (\overline{x:\pi})\:\spec : \pi_{res}}}\\[1em]

      \inferrule*[Left=(SValGhost)]
      {\neg\textit{is\_ghost}(\atts) \\ \textit{kind}(\atts) = \kind \\\\
      \translate{\atts'}{\textit{function spec}}{\spec} \\
      \translate{\pi,\atts'}{\textit{function args}}{\overline{x:\pi'},\pi_{res}}}
      {\translate{\mathtt{val}\: \atts \: f : \pi \: \atts'}{signature}
      {\mathtt{val}\:\kind\:\ghost f (\overline{x:\pi})\:\spec : \pi_{res}}}\\[1em]

      \inferrule*[Left=(SType)]
      {\translate{td_0}{\textit{type definition}}{td_0'} \\
      \translate{\overline{td_1}}
      {\textit{type definition}}{\overline{td_1'}}}
      {\translate{\texttt{type} \; td_0  \; \overline{\texttt{and}\; td_1}}
      {signature}
      {\texttt{type}\: td_0' \; \overline{\texttt{with}\:td_1'}}}
    \end{array}
  \]
  \end{judge}
  \caption{Translation of \ocaml signatures into \whyml.}
  \label{fig:sigs}
\end{figure}

\paragraph{Programs.} An \ocaml program is simply a list of top-level
declarations. These are translated into a \whyml module, as follows:
%
\[
  \inferrule*[Left=(Program)]
  {\translate{\overline{d}}{declaration}{\overline{d'}}}
  {\translate{\overline{d}}{program}
    {\mathtt{module}\,\mathcal{M}\;\overline{d'}\;\mathtt{end}}}
\]
%
The name~$\mathcal{M}$ of the generated module is issued from the \ocaml file
that contains the original program. If file \texttt{foo.ml} contains the
program~$p$, it gets translated into \texttt{module\:Foo\;$p$\;end}. In summary,
a generated \whyml program contains a single module, which represents the
top-level module of an \ocaml file. In turn, each sub-module is translated into
a \whyml scope.

% \bibliographystyle{plain}
% \bibliography{local}

\end{document}

\section{Verified implementations of queue data structures}
\label{sec:verif-impl-prior}

\begin{figure}
  \centering
  \begin{gospel}
  type 'a t = {
    self : 'a list * 'a list;
    view : 'a list [@ghost];
  } (*@ invariant let (prefix, xiffus) = self in (prefix=[] -> xiffus=[]) &&
          view = prefix @ List.rev xiffus *)

  let empty = { self = [], []; view = [] }
  (*@ t = empty ensures t.view = [] *)

  let [@logic] is_empty (q: 'a t) = match q.self with
    | [], _ -> true
    | _ -> false
  (*@ b = is_empty q ensures b <-> q.view = [] *)

  let add queue elt = match queue.self with
    | [], [] ->
        { self = [elt], []; view = [elt] }
    | prefix, xiffus ->
        { self = prefix, elt :: xiffus; view = queue.view @ [elt] }
  (*@ r = add queue elt ensures r.view = queue.view @ (elt :: []) *)

  let head (param: 'a t) = match param.self with
    | head :: _, _-> head
    | [], _ -> raise Not_found
  (*@ x = head param
        raises  Not_found -> is_empty param
        ensures match param.view with [] -> false | y :: _ -> x = y *)

  let [@logic] [@ghost] tail_list = function
    | [] -> assert false
    | _ :: l -> l
  (*@ r = tail_list param
        requires param <> []
        ensures  match param with [] -> false | _ :: l -> r = l *)

  let tail t = match t.self with
    | [_], xiffus ->
        { self = List.rev xiffus, []; view = tail_list t.view }
    | _ :: prefix, xiffus ->
        { self = prefix, xiffus; view = prefix @ List.rev xiffus }
    | [], _ -> raise Not_found
  (*@ r = tail t
        raises  Not_found -> is_empty t
        ensures r.view = tail_list t.view *)
  \end{gospel}
  \caption{Verified implementation of a purely applicative queue data
    structure.}
  \label{fig:app-queue}
\end{figure}

\begin{figure}
  \centering
  \begin{gospel}
  type 'a t = {
    mutable front: 'a list;
    mutable rear : 'a list;
    mutable view : 'a list [@ghost];
  } (*@ invariant (front = [] -> rear = []) && view = front ++ List.rev rear *)

  let create () = { front = []; rear  = []; view  = []; }
  (*@ q = create () ensures q.view = [] *)

  let [@logic] is_empty q = match q.front with
    | [] -> true
    | _ -> false
  (*@ b = is_empty q ensures b <-> q.view = [] *)

  let push x q =
    if is_empty q then q.front <- [x] else q.rear <- x :: q.rear;
    q.view <- q.view @ (x :: [])
  (*@ push x q ensures q.view = (old q.view) @ (x :: []) *)

  let [@ghost] head_list = function
    | [] -> assert false
    | x :: _ -> x
  (*@ r = head_list param
        requires param <> []
        ensures  match param with [] -> false | y :: _ -> r = y *)

  let [@ghost] tail_list = function
    | [] -> assert false
    | _ :: l -> l
  (*@ r = tail_list param
        requires param <> []
        ensures  match param with [] -> false | _ :: l -> r = l *)

  let pop q = match q.front with
    | [] -> raise Not_found
    | [x] ->
        q.front <- List.rev q.rear; q.rear  <- []; q.view  <- tail_list q.view;
        x
    | x :: f ->
        q.front <- f; q.view  <- tail_list q.view;
        x
  (*@ x = pop q
        raises  Not_found -> is_empty (old q)
        ensures x :: q.view = (old q).view *)

  let transfer q1 q2 =
    let [@ghost] done_view = ref [] in
    while not (is_empty q1) do
      (*@ variant   q1.view *)
      (*@ invariant (q1.front = [] -> q1.rear = []) /\
                    (q2.front = [] -> q2.rear = []) *)
      (*@ invariant q1.view = q1.front @ List.rev q1.rear /\
                    q2.view = q2.front @ List.rev q2.rear *)
      (*@ invariant old q1.view = !done_view @ q1.view *)
      (*@ invariant q2.view = (old q2.view) @ !done_view *)
      done_view := !done_view @ (head_list q1.view :: []);
      push (pop q1) q2
    done
  (*@ transfer q1 q2
        raises   Not_found -> false
        ensures  q1.view = []
        ensures  q2.view = (old q2.view) @ (old q1.view) *)
  \end{gospel}
  \caption{Ephemeral queue data structure.}
  \label{fig:mut-queue}
\end{figure}

\section{Formally verified Leftist Heaps}
\label{sec:form-verif-left}

\begin{gospel}
module type TOTAL_PRE_ORD = sig
  type t

  (*@ function le : t -> t -> bool *)

  (*@ axiom reflexive : forall x. le x x *)
  (*@ axiom total     : forall x y. le x y \/ le y x *)
  (*@ axiom transitive: forall x y z. le x y -> le y z -> le x z *)

  val leq : t -> t -> bool
  (*@ b = leq x y
        ensures b <-> le x y *)
end

module Make(E : TOTAL_PRE_ORD) = struct
  type elt = E.t

  type t =
    | E
    | N of int * elt * t * t

  (*@ function rank (h: t) : integer = match h with
        | E -> 0
        | N _ _ l r -> 1 + min (rank l) (rank r) *)

  (*@ predicate leftist (h: t) = match h with
        | E -> true
        | N n _ l r ->
            n = rank h && leftist l && leftist r && rank l >= rank r *)

  let[@logic] rec size = function
    | E -> 0
    | N (_,_,l,r) -> 1 + size l + size r
  (*@ r = size param
        ensures 0 <= r
        ensures r = 0 <-> param = E *)

  (*@ function occ (x: elt) (h: t) : integer = match h with
        | E -> 0
        | N _ e l r -> let occ_lr = occ x l + occ x r in
            if x = e then 1 + occ_lr else occ_lr *)

  let [@lemma] rec occ_nonneg (y: elt) = function
    | E -> ()
    | N (_, _, l, r) -> occ_nonneg y l; occ_nonneg y r
  (*@ occ_nonneg y param
        ensures 0 <= occ y param *)

  (*@ predicate mem_heap (x: elt) (h: t) =
        0 < occ x h *)

  (*@ predicate le_root (e: elt) (h: t) = match h with
        | E -> true
        | N _ x _ _ -> E.le e x *)

  let [@lemma] le_root_trans (x: elt) (y: elt) = function
    | E -> ()
    | N (_, _, _, _) -> ()
  (*@ le_root_trans x y param
        requires E.le x y
        requires le_root y param
        ensures  le_root x param *)

  (*@ predicate is_heap (h: t) = match h with
        | E -> true
        | N _ x l r -> le_root x l && is_heap l && le_root x r && is_heap r *)

  (*@ function minimum (h: t) : elt *)
  (*@ axiom minimum_def: forall l x r n. minimum (N n x l r) = x *)

  (*@ predicate is_minimum (x: elt) (h: t) =
        mem_heap x h && forall e. mem_heap e h -> le x e *)

  let [@lemma] rec root_is_miminum = function
    | E -> assert false
    | N (_, _, l, r) ->
        begin match l with E -> () | _ -> root_is_miminum l end;
        match r with E -> () | _ -> root_is_miminum r
  (*@ is_minimum param
       requires is_heap param && 0 < size param
       ensures  is_minimum (minimum param) param
       variant  param *)

  (*@ predicate leftist_heap (h: t) =
        is_heap h && leftist h *)

  let empty = E
  (*@ r = empty
        ensures r = E *)

  let[@logic] is_empty = function
    | E -> true
    | N (_, _, _, _) -> false
  (*@ b = is_empty param
        ensures b <-> param = E *)

  exception Empty

  (* Rank of the tree *)
  let _rank = function
    | E -> 0
    | N (r, _, _, _) -> r
  (*@ r = _rank param
        requires leftist_heap param
        ensures  r = rank param *)

  let _make_node x a b =
    if _rank a >= _rank b
    then N (_rank b + 1, x, a, b)
    else N (_rank a + 1, x, b, a)
  (*@ h = _make_node x a b
        requires leftist_heap a && leftist_heap b
        requires le_root x a && le_root x b
        ensures  leftist_heap h
        ensures  minimum h = x
        ensures  size h = 1 + size a + size b
        ensures  occ x h = 1 + occ x a + occ x b
        ensures  forall y. x <> y -> occ y h = occ y a + occ y b *)

  let rec merge t1 t2 =
    match t1, t2 with
      | t, E -> t
      | E, t -> t
      | N (_, x, a1, b1), N (_, y, a2, b2) ->
        if E.leq x y
        then _make_node x a1 (merge b1 t2)
        else _make_node y a2 (merge t1 b2)
  (*@ h = merge t1 t2
        requires leftist_heap t1 && leftist_heap t2
        ensures  size h = size t1 + size t2
        ensures  forall x. occ x h = occ x t1 + occ x t2
        ensures  leftist_heap h
        variant  size t1 + size t2 *)

  let insert x h =
    merge (N(1,x,E,E)) h
  (*@ new_h = insert x h
        requires leftist_heap h
        ensures  leftist_heap new_h *)

  let add h x = insert x h
  (*@ new_h = insert x h
        requires leftist_heap h
        ensures  leftist_heap new_h *)

  let rec filter p = function
    | E -> E
    | N (_, x, l, r) ->
        if p x then _make_node x (filter p l) (filter p r)
        else merge (filter p l) (filter p r)
  (*@ h = filter p param
        requires leftist_heap param
        variant  param
        ensures  leftist_heap h
        ensures  forall x. not (p x) -> occ x h = 0
        ensures  forall x. p x -> occ x h = occ x param *)

  let find_min_exn = function
    | E -> raise Empty
    | N (_, x, _, _) -> x
  (*@ r = find_min_exn param
        requires leftist_heap param
        raises   Empty -> is_empty param
        ensures  r = minimum param *)

  let find_min = function
    | E -> None
    | N (_, x, _, _) -> Some x
  (*@ r = find_min param
        requires leftist_heap param
        ensures  match r with
                 | None -> is_empty param
                 | Some x -> is_minimum x param *)

  let take = function
    | E -> None
    | N (_, x, l, r) -> Some (merge l r, x)
  (*@ r = take param
        requires leftist_heap param
        ensures  match r with
                 | None -> is_empty param
                 | Some (h, x) ->
                     is_minimum x param && (* minimum element *)
                     occ (minimum param) h = occ (minimum param) param - 1 &&
                     forall y. y <> minimum param -> occ y param = occ y h &&
                     size h = size param - 1 &&
                     leftist_heap h *)

  let rec delete_all eq x = function
    | E -> E
    | N (_, y, l, r) as h ->
        if eq x y then merge (delete_all eq x l) (delete_all eq x r)
        else if E.leq y x then
          _make_node y (delete_all eq x l) (delete_all eq x r)
        else h
  (*@ h = delete_all eq x param
        requires leftist_heap param
        requires forall a b. a = b <-> eq a b
        variant  param
        ensures  leftist_heap h
        ensures  occ x h = 0
        ensures  forall y. x <> y -> occ y h = occ y param
        ensures  size h = size param - occ x param *)
end
\end{gospel}

\end{document}


Heap operations:
* _make_node
* merge
* insert (uma linha e chama o merge) X
* find_min_exn                       X
* filter

(juntar frase que descreve rapidamente o comportamento do insert e do find_min)
